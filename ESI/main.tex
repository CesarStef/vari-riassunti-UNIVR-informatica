\documentclass{article}
\usepackage[utf8]{inputenc}
\usepackage[italian]{babel}

%\title{Teoria Sicurezza}
%\author{cattonarstefano95 }
%\date{June 2019}

\begin{document}

    \begin{itemize}
        \item \textbf{Segnale}: Una qualsiasi grandezza che varia in un dominio in maniera deterministica o aleatoria e che trasporta informazioni
        \item \textbf{Rumore}: Tutto ciò che è associato al segnale, ma non porta informazione. Disturba la ricezione del segnale e l'estrazione dell'informazione
        \item \textbf{Energia di un segnale}:Attributo associato ad un segnale espresso in joule. Per trasmettere un segnale mi serve energia
        \item \textbf{Potenza di un segnale}: L'energia per unità di tempo
        \item \textbf{SNR o Signal to Noise Ratio}: Rapporto tra la potenza del segnale e la potenza del rumore
        \item \textbf{Sistema di elaborazione}: modello  matematico che a uno o piu segnali di ingresso reagisce producendo lo stesso numero di segnali di uscita{\Large FaRE I DISEGNETTI}
        Un sistema agisce su un segnale:
        \begin{itemize}
            \item Migliorando il rapporto segnale/rumore
            \item estrando informazioni dal segnale
        \end{itemize}
        \item \textbf{Immagine}: E' un segnale in 2 dimensioni di durata o estensione FINITA.
        \item Un segnale può essere visto attraverso il suo sviluppo temporale [f(t)] oppure attraverso quello frequenziale [f(u)]
        \item Il dominio frequenziale conta quante volte in un unità di tempo si ripete un determinato evento. Più la frequenza è alta maggiore è il ripetersi di un evento nel tempo
        \item Anche le immagini hanno un dominio frequenziale. In un immagine in scala di grigi (0-255 per pixel) la frequenza conta il ripetersi di un determinato grigio.
        \item \textbf{Filtraggio}: Trasformazione di un segnale da una forma ad un altra
        \item \textbf{Acquisizione/ricezione}: Più è alta la frequenza più è costoso e difficile acquisire e ricevere un segnale
        \item \textbf{Campionamento}: I segnali acquisiti in natura sono analogici, per trasformarli in digitali serve campionarli.
        Il campionamento trasforma un segnale continuo(analogico) in un segnale discreto(digitale).
        Un campionamento errato genera perdita di informazione rispetto al segnale originale. Questo problema si chiama aliasing.
        \item \textbf{Filtraggio Lineare:} Il filtraggio lineare di un immagine viene ottenuto facendo la convoluzione tra l'immagine con una matrice anche chiamata maschera.
    \end{itemize}
%\maketitle

\end{document}
